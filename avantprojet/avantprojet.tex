\documentclass[a4paper,12pt]{report}
\usepackage[utf8]{inputenc}
\usepackage[francais]{babel}
\usepackage[top=2cm, bottom=2cm, left=2cm, right=2cm]{geometry}
\title{Avant-projet du projet informatique}
\author{Matthieu Denoux - Groupe 1}


\begin{document}
\maketitle

%%%%%%%%%%%%%%%%%%%%%%%%%%%%%%%%%%
%%                              %%
%%   ------------------------   %%
%%                              %%
%%%%%%%%%%%%%%%%%%%%%%%%%%%%%%%%%%
\chapter{Présentation du sujet}

%%%%%%%%%%%%%%%%%%%%%%%%%%%%%%%%%%
%%   ------------------------   %%
%%%%%%%%%%%%%%%%%%%%%%%%%%%%%%%%%%
\section{Principe général}
\paragraph{}
Le sujet choisi a pour intitulé \textbf{Babyfoot en réseau}. Il s'agit de concevoir un système complet de jeu en réseau.
Le système serait donc séparé en deux parties, un serveur et un client. Chaque joueur pourrait donc se connecter à une partie
n'ayant pas encore commencé et une fois le nombre de joueurs réunis (\emph{2 ou 4}), la partie serait lancée. Il faut donc réaliser à la fois 
le système réseau, l'interface graphique et imaginer un gameplay qui rende le jeu agréable.

%%%%%%%%%%%%%%%%%%%%%%%%%%%%%%%%%%
%%   ------------------------   %%
%%%%%%%%%%%%%%%%%%%%%%%%%%%%%%%%%%
\section{L'interface graphique}
\paragraph{}
Au niveau de l'interface graphique, il faudra réaliser plusieurs fenêtres successives de menus pour parvenir jusqu'au jeu lui-même. Ces fenêtres seront réalisées avec la bibliothèque inclue dans le package standard \textbf{Swing}. J'ai aussi choisi d'utiliser des JFrame et donc de programmer une application à part entière et non une appliquette.
\paragraph{Menu principal}
Un premier menu, dit \emph{menu principal}, permettra de commencer une nouvelle partie en ligne, de rejoindre une partie
déjà en cours, de modifier les options ou bien de quitter le jeu. Un header sensé unifier les différentes fenêtres sera aussi présent, il donnera en plus une certaine identité graphique au jeu.
\paragraph{Commencer une partie}
Si l'on commence une nouvelle partie en ligne, on se retrouve dans une \og salle d'attente \fg. Là, il est possible de configurer la partie que l'on souhaite lancer, l'ouvrir à d'autres joueurs puis d'attendre que d'autres joueurs rejoignent la partie. Une fois que les équipes sont complètes, on peut lancer le jeu.
\paragraph{Rejoindre une partie}
On peut aussi sélectionner une partie dans la liste des parties déjà commencées, dans la limite des places disponibles. On rejoint alors un salon similaire à celui décrit dans la section précédente où l'on attend que la partie soit complète avant que le meneur, celui qui a créé la partie, ne lance le jeu.
\paragraph{Déroulement d'une partie}
Il y aurait deux gameplays différents : 
\begin{itemize}
    \item l'un entièrement manuel verrait le joueur maître de ses possibilités. Ainsi, les touches \og A \fg, \og Z \fg, \og E \fg, \og R \fg permettraient de sélectionner la canne que l'on souhaiterait manier. Les flèches \og Haut \fg et \og Bas \fg permettraient quant à elle de déplacer les cannes tandis que la barre espace commanderait le tir. Je ne pense pas réaliser plusieurs puissances de tir possibles mais cela pourrait consister en une amélioration de la richesse du gameplay.
    \item l'autre, principalement rencontrée dans les autres jeux de babyfoot trouvés en ligne, serait plus automatisée : les cannes se déplaceraient toutes ensemble de manière synchronisée avec les flèches \og Haut \fg et \og Bas \fg. Le tir pourrait être ou non automatisé selon que j'aurais le temps pour configurer cette fonctionnalité supplémentaire.
\end{itemize}
\paragraph{Options}
Un petit menu sera consacré aux options, notamment le type de jeu que l'on souhaite utiliser (manuel ou automatisé, auquel cas jusqu'à quel point) et d'autres options.


%%%%%%%%%%%%%%%%%%%%%%%%%%%%%%%%%%
%%                              %%
%%   ------------------------   %%
%%                              %%
%%%%%%%%%%%%%%%%%%%%%%%%%%%%%%%%%%
\chapter{Analyse de la solution envisagée}


%%%%%%%%%%%%%%%%%%%%%%%%%%%%%%%%%%
%%   ------------------------   %%
%%%%%%%%%%%%%%%%%%%%%%%%%%%%%%%%%%
\section{Découpage en modules}
\paragraph{}
Je prévois de découper le code en deux grandes parties :
\begin{itemize}
    \item Partie client
    \item Partie serveur
\end{itemize}
Les deux parties (client/serveur) seront elles-mêmes séparées en plusieurs modules.
\begin{itemize}
    \item L'interface graphique (rangée dans /gui) qui sera absente du serveur
    \item La partie réseau de l'application (rangée dans /network), responsable de l'émission et de la réception des requêtes.
    \item Le c\oe{}ur algorithmique de l'application (rangée dans /core) contenant les classes gérant les matchs, les joueurs, etc.
\end{itemize}
Chaque partie et chaque module comporteront donc plusieurs classes qui se chargeront chacune d'une des tâches du module parent.


%%%%%%%%%%%%%%%%%%%%%%%%%%%%%%%%%%
%%   ------------------------   %%
%%%%%%%%%%%%%%%%%%%%%%%%%%%%%%%%%%
\section{L'interface graphique}
\paragraph{MainFrame.java}
Gère la fenêtre qui englobe tout le reste. On utilisera en fait des JPanel pour modifier le contenu de cette fenêtre. J'empêcherai dans un premier temps de modifier la taille de la fenêtre pour éviter d'avoir des problèmes de dessin du terrain de babyfoot à gérer. Cette classe contient le \textbf{main} du programme.
\paragraph{BPanel.java}
Classe abstraite présentant certaines actions mécaniques quant aux caractéristiques des JPanel utilisés (header, background color, taille, layoutmanager, etc.). La plupart des panels ci-dessous en héritent.
\paragraph{ConnectPanel.java}
Gère le premier écran affiché à l'ouverture du jeu. On y choisit le pseudonyme que l'on souhaite utiliser dans le reste du jeu et qui s'affichera ensuite sur les écrans lors du contact avec d'autres joueurs.
\paragraph{MenuPanel.java}
Gère le menu principal affiché après connexion. Contient les boutons qui mèneront vers les principales actions citées plus haut (nouvelle partie, rejoindre, options, quitter).
\paragraph{NewPanel.java}
Lorsque l'on lance une nouvelle partie, on obtient cet écran qui contiendra les principales options nécessaires pour configurer une partie puis l'ouvrir à des joueurs extérieurs. Une fois cette partie ouverte, on aboutit à une \emph{salle d'attente}.
\paragraph{ServersPanel.java}
Gère la liste des parties actuellement en recherche de joueurs lorsque l'on cherche à rejoindre une partie. On sélectionne une partie dans cette liste puis on aboutit à une \emph{salle d'attente}.
\paragraph{WaitingRoomPanel.java}
Une fois la partie configurée et ouverte, on aboutit dans cette salle d'attente qui attendra que certaines conditions soient réunies pour permettre au jeu d'être commencé. On peut \emph{aussi} y accéder depuis la partie \emph{rejoindre une partie}.
\paragraph{GamePanel.java}
La partie à proprement parler. Donc le conteneur du dessin du terrain qui gère les éléments extérieurs, les événements et toute autre interaction avec le reste du code. Inclut la zone de dessin décrite ci-dessous.
\paragraph{GameZone.java}
Zone de dessin chargée de représenter le terrain, les joueurs, etc.
\paragraph{ChatPanel.java}
Une partie un peu à part qui est chargée de gérer l'affichage du chat. Celui-ci sera présent à de nombreux endroits : la liste des parties disponibles, la salle d'attente, la création d'une partie et durant le jeu lui-même. Or, tous ces affichages sont centralisés dans cet unique fichier qui centralise ainsi le traitement.


%%%%%%%%%%%%%%%%%%%%%%%%%%%%%%%%%%
%%   ------------------------   %%
%%%%%%%%%%%%%%%%%%%%%%%%%%%%%%%%%%
\section{Le réseau}
\subsection{Le principe général}
\paragraph{}
Des messages sont échangés entre le client et le serveur et ce de façon répétée. Il faut donc utiliser des Thread pour gérer l'émission et la réception de données depuis le serveur et un Thread pour gérer la réception côté client. Le serveur se chargera ensuite d'enregistrer les données statiques devant être conservées.
\subsection{La partie client}
\paragraph{Main.java}
Il s'agit du main du client qui initialise les fenêtres et les différentes classes utilisées.
\begin{quote}
	private static Player player;
    
	private static Chat chat;
    
	private static Client client;
\end{quote}
\paragraph{Client.java}
Cette classe gère la connexion au serveur : il y a en tout trois connexions (une pour le tchat, une pour les joueurs et une pour les données des matchs) afin de pouvoir envoyer plusieurs requêtes simultanément du même client. Toutes les requêtes envoyées sont sous la forme de chaîne de caractères avec l'utilisation de \og - \fg comme séparateurs.
\begin{quote}
	private static Socket socketChat;
    
	private static Socket socketPlayer;
    
	private static Socket socketMatch;
    
	private ChatClient cc;

	private PlayerClient pc;

	private MatchClient mc;

	private Thread tChat;

	private Thread tPlayer;
    
	private Thread tMatch;
\end{quote}
\paragraph{PlayerClient.java}
Gère les différentes actions possibles par et sur les joueurs et les requêtes envoyées au serveur pour pouvoir satisfaire le système (connexion d'un joueur, déconnecter un joueur, ajouter un match, etc.). Elle contient aussi un Thread qui écoute l'entrée de la socket.
\paragraph{MatchClient.java}
Procède de la même façon que PlayerClient pour les matchs, envoie les requêtes au serveur et 
\paragraph{ChatClient.java}
S'occupe de gérer la partie client du tchat : une fois instanciée par le ChatPanel, elle permet d'envoyer des données au serveur et de récupérer la liste des messages disponibles dans le salon. Elle permet aussi de changer de salon, d'afficher la liste des salons et celle des joueurs connectés sur le jeu.
\subsection{La partie serveur}
\paragraph{Server.java}
Gère les différentes requêtes et les redirige vers les autres entités du serveur (chat, jeu, etc.). Contient un système de Thread pour pouvoir gérer simultanément ces différentes actions, un Thread général qui gère les requêtes.
\begin{quote}
    ServerSocket socketserver = null;
    
	Socket socket = null;
    
	BufferedReader in;
    
	PrintWriter out;
    
	String login;
    
	public static ChatServer tchat;
    
	public static PlayerServer tplayer;
    b
	public static MatchServer tmatch;
\end{quote}
\paragraph{PlayerServer.java}
Gère la connexion au serveur d'un des joueurs. L'enregistre dans une liste de joueurs connectés. Peut aussi renvoyer la liste des joueurs connectés actuellement et leur état.
\paragraph{MatchServer.java}
Gère tous les échanges de données sur les matchs. S'occupe ensuite de fournir les informations nécessaires au client sur l'état actuel de la position de la balle par exemple. Nécessitera peut-être, pour des questions de réactivité, d'être multi-threadé.
\paragraph{ChatServer.java}
S'occupe de gérer la partie serveur du tchat : écoute un port, récupère les messages envoyés par des joueurs ainsi que le salon où ils se trouvaient à ce moment là. Les enregistre dans un fichier ou bien dans une base de données (à voir).


%%%%%%%%%%%%%%%%%%%%%%%%%%%%%%%%%%
%%   ------------------------   %%
%%%%%%%%%%%%%%%%%%%%%%%%%%%%%%%%%%
\section{Le c\oe{}ur algorithmique}
\paragraph{Player.java}
Le bloc de base du c\oe{}ur algorithmique est le joueur. Il est instancié par le serveur et a plusieurs attributs : son login, son état actuel (s'il est en train de jouer ou non). Si le joueur est en match, la classe contient aussi les barres qu'il a le droit de déplacer.
\begin{quote}
	private Match match;
    
	private String login;
\end{quote}
\paragraph{Match.java}
Contient les données principales pour une partie, notamment l'avancement de la partie, l'état des scores, les joueurs y participant, etc. Sera appelée par le serveur. Cette classe contient aussi toutes les informations sur les données factuelles d'une partie, à savoir la position de la balle, la position des barres, etc.
\begin{quote}
    private int leftScore;
    
	private int rightScore;
	
	private enum Types = { ONEVSONE, TWOVSTWO, ONEVSTWO };
	
	private enum States = { WAITING, FULL, PLAYING, FINISHED };  
	
	private Types type;
	
	private States state;
	
	private Player player1;
    
	private Player player2;
    
	private Player player3;
    
	private Player player4;
\end{quote}
\paragraph{Utils.java}
Avoir une classe abstraite qui gère les fonctions utilitaires comme la mise-en-forme de la date, la mise-en-forme des requêtes, les fonctions de hash utilisées pour vérifier la validité des requêtes, etc.
\paragraph{Database.java}
Gère la base de données PostGreSQL ou bien textuelle qui pourrait être utilisée par le programme pour stocker des informations. A priori, seul le serveur interviendra sur la base de données et y entrera les informations relatives à l'historique des tchats et des matchs. Peut être utile d'entrer directement depuis les clients les données dans la base sans passer au préalable par le serveur (exemple : connexion d'un joueur, tchat, etc.). L'intérêt est de n'avoir qu'une seule connexion.

%%%%%%%%%%%%%%%%%%%%%%%%%%%%%%%%%%
%%                              %%
%%   ------------------------   %%
%%                              %%
%%%%%%%%%%%%%%%%%%%%%%%%%%%%%%%%%%
\chapter{Échéancier}
\begin{enumerate}
    \item \textbf{Fin Décembre} : rédiger l'avant-projet.
    \item \textbf{Début/Mi Janvier} : obtenir la validation et les annotations sur la structure choisie.
    \item \textbf{Fin Janvier} : réaliser la partie graphique du programme et avoir regardé les grandes lignes du développement serveur / bases de données. Avoir mis au point les éléments de base du gameplay (interaction joueur/machine).
    \item \textbf{Fin Février} : développer le serveur et la gestion des différents types de requêtes. Mettre au point le tchat et la gestion des joueurs avec la base de données.
    \item \textbf{Mi Mars} : dresser les liens entre serveur et jeu. Tester.
\end{enumerate}
\end{document}          
