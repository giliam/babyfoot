\documentclass[a4paper,12pt]{report} % Rapport de MIG SE 2013
\usepackage[a4paper, margin=2cm]{geometry}

\usepackage[T1]{fontenc}
\usepackage[utf8]{inputenc}
\usepackage[francais]{babel}

% Title Page
\title{Avant-projet du projet informatique}
\author{Matthieu Denoux}


\begin{document}
\maketitle
\tableofcontents
\chapter{Présentation du sujet}
\section{Principe général}
\paragraph{}
Le sujet choisi a pour intitulé \textbf{Babyfoot en réseau}. Il s'agit de concevoir un système complet de jeu en réseau.
Le système serait donc séparé en deux parties, une serveur et une client. Chaque joueur pourrait donc se connecter à une partie
n'ayant pas encore comméncé et une fois le nombre de joueurs réunis (\emph{2 ou 4}), la partie serait lancée. Il faut donc réaliser à la fois 
le système réseau, l'interface graphique et imaginer un gameplay qui rende le jeu agréable.
\section{L'interface graphique}
\paragraph{}
Au niveau de l'interface graphique, il faudra réaliser plusieurs fenêtres successives de menus pour parvenir jusqu'au jeu lui-même.
\subsection{Menu principal}
\paragraph{}
Un premier menu, dit \emph{menu principal}, permettra de commencer une nouvelle partie en ligne, de rejoindre une partie
déjà en cours, de modifier les options ou bien de quitter le jeu.
\subsection{Commencer une partie}
\paragraph{}
Si l'on commence une nouvelle partie en ligne, on se retrouve dans une \og salle d'attente \fg. Là, on attend que d'autres joueurs
rejoignent la partie. 
\subsection{Rejoindre une partie}
\paragraph{}
\subsection{Options}
\paragraph{}
\end{document}          
